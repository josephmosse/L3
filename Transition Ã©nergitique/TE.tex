\documentclass{article}
\usepackage[utf8]{inputenc}
\usepackage{multicol}
\usepackage{array}
\usepackage{upgreek}
\usepackage{amsmath}
\usepackage[dvipsnames]{xcolor}
\usepackage{geometry}

\title{Transition énergétique}

\begin{document}
	\maketitle
	
\section{Pourquoi étudier le choc pétrolier}
Première fois que l'humanité a essayé de sortir des énergies fossiles. Avant les années 60 le pétrole est géré par le cartel des 7 sœurs. \\
Dans les années 50,l'URSS rentre sur le marché du pétrole $ \Rightarrow $ Les 7 soeurs font baisser le prix du baril de pétrole pour concurrencer l'URSS. \\
En 1960 un autre cartel se forme (OPEP = Venezuela, Arabie Saoudite ) en réaction du cartel d'utilisateur des ressources(7 soeurs). En 1968, la Lybie rentre dans le OPEP et menace de fournir du pétrole, le cartel des 7 soeurs cède a la pression de kadhaffi. 
\subsection{$1^{er}$ Choc pétrolier}
Multiplication par 3 du prix du baril (la demande de pétrole est très inélastique au prix). Premier choc qui affecte les pays développés. Ce choc pétrolier pousse les états à rechercher une alternative énergétique, les premières lois de transition énergétique apparaissent  : 
\begin{itemize}
	\item Programme nucléaire en France
	\item Maitrise de la demande d'énergie au Japon
	\item premieres politique de soutien aux ENR dans de nombreux pays
\end{itemize}
Remarques sur la consommation d'énergie
\begin{itemize}
	\item La consommation d'énergie est fortement inélastique au prix
	\item La part des dépenses d'énergie dans la consommation finale des ménages français est restée stable.
\end{itemize}

\subsubsection{Effet du choc pétrolier sur l'efficacité des véhicules}
Les USA se mettent a réguler la production des véhicules pour améliorer l'efficacité des véhicules (faire rouler plus pour moins) $ \Rightarrow $ Innovation rapide. 

\section{Les énérgies : utilisations et caractéristiques techniques}
\subsection{Vue d'ensemble }
\subsubsection{Mixes énergétique et électrique Français}
La majorité de l'électricité en france est produite par le nucléaire, hydrolique et peu par les énergies renouvelables .\\
\subsubsection{Répartition sectorielle de la consommation d'énergie en France}
Plus gros de la consommation en France: 
\begin{itemize}
	\item L'immobilier (maisons)
	\item Véhicules
	\item Industrie
\end{itemize}

\subsection{•}

\subsubsection{Les marchés de l'énergie}
Le marché se définit par : 
\begin{itemize}
	\item La nature / qualité du bien 
	\item Le lieu de l'échange (si le coût de transport > 0)
	\item Le moment de mise à disposition du bien échangé
	\item[$ \Rightarrow $] Segmentation des marchés de l'énergie = multiplicité des marchés
\end{itemize}
L'énergie est un bien stratégique et de première nécessite (dont on ne peut pas se passer pour son économie). Donc forte implication des pouvoirs publics. \\
Production très intensive en capital et durée de vie des équipements très longue ( $ > 20 $ ans). \\
Existence de rentes et de pouvoir de marché.

\textbf{Marché du pétrole} : Mondialement unifié (tous les pays sont en concurrence), \( +50 \) de la production est échangée a l'échelle mondiale. Sa rentabilité est très grande \( P_p >> Cm \) . Coûts de transport faibles \( 1.5 \) , coûts de stockage faibles (sauf lorsque les réserves sont saturées).

\textbf{Gaz naturel} : 3 aires géographiques (Europe, Amérique, Asie), 3 formes de stockage du gaz (gazeux , liquide schiste). \( 25 \) de la production est mondialement engagée. Son prix est indexé sur le prix du pétrole (substitut) \( P_g = F(P_p) \). Coûts de transport et de stockage cher .

\textbf{Charbon} : Segmentation géographique égale (partout dans le monde) . Peu rentable \( P_c  = Cm \). Coûts de transport et de stockage faibles

\textbf{Uranium} : Marché secondaire (contrats bilatéraux et spot) , le processus de raffinage est très important. 

\textbf{Electricité} : Segmentation régionale, le réseau de transport est très important. Les prix sont fonction des énergies primaires

\subsubsection{Les rentes dans les secteurs de l'énergie}
\textbf{Rente :} Profit que le producteur va dégager.\\
\textbf{Rente Ricardienne :} plusieurs producteurs vendent le même produit et qui disposent de technologies de production différentes.

Rente d'épuisabilité ou rente de Hotelling : prise en compte du coût d'opportunité à vendre tout de suite 


\subsection{Matrices swot}
\subsubsection{Matrice SWOT énergie éolienne}
\subsubsection{Matrice SWOT énergie solaire PV}
	Forces  Faiblesses \\
	- pas d'émission directes de GES  	- Couts encore très élevés
	- pas de combustibles  	- intermittence de la production
	- Faible couts  	- filière de production énergivore
	- Forte mudulatrité 
	- Complémentarité avec d'autres secteurs
	
	
	- Plus grand gisement d'energie renouvelable
	- Opportunité de repowering
	- Baisse rapide des couts ces deux dernires décénies
	- nouvelles technologies
	
	- Marché très compétitif (risques pour l'investissement)
	- importance du stockage
	- Rendement qui décroit avec la température	


\subsubsection{Matrice SWOT charbon}
Forces
reserves et ressources abondantes
repartition géographique relativement équitable
combustible peu couteux
production pilotable

Faiblesses
Forte pollutions globales
Fortes pollutions locales
Exploitation de la ressource dangereuse

Opportunités
Couplage avec le carbon caputre and storage
Combustion par le "lit fluidisé " > moins de soufre
amélioration des rendements

\subsubsection{GAZ}
Moins emetteur que le charbon

faiblesses
transport couteux
concentration géographique des réserves

opportunités 
abondance des ressouces non conventionnelles
energie fossile la moins polluante


\subsubsection{Nucléaire}
Electricité peu couteuse
repartition homogene
electricité décarbonée 

faiblesses
maitrise de la technologie
dépendance aux filières de raffinage
risque d'accidents 

opportunités
diffusuin des small modulars reactos
fusion nucléaire
attrait des pays avec faibles normes de sécurité
reacteurs 4eme generation

menaces
fusion nucléaires très éloignés de la maturité commerciale
couts croissant dans le temps
cout de démentelement
traitement des dechets
risque de diffusion de l'arme atomique

\subsubsection{Biomasse}
forces 
energie renouvellable (si renouvellement)
applicaitons miltiples
couplage avec le charbon

faiblesse
Conflit d'usage avec l'alimentation
conflits locales si installations anciennes
conflit d'usage entre les services énéergtiques
pollutions locales (méthanisation)

opportunités
3eme generation
co generation
recupération et distribution du biogaz

\subsubsection{Hydro}
electricité décarbonnée
production pilotable
production de base
moyenne de stockage

faiblesses
impacts négatifs sur la faune locale
risque d'accident
faible exploitation de gisement


\section{La raréfaction des ressources fossiles}
\subsection{Réserves et ressources}
Réserves  : dépots ayant été découverts et évalués et ont estime qu'aux conditions de marché en vigueur, leur exploitation est économiquement viable \\
Ressources : Réserves + dépôts découverts et potentiellement profitables + dépôts non découverts mais prédits sur la base des études géologiques \\
Différentes mesures des ressources (prouvées, probables, possibles, non découvertes) \\ 
Les réserves d'hydrocarbures ne sont pas distribués de façon homogène.
ex : 
\begin{itemize}
	\item Amérique centrale et sud , pétrole = Vénézuela 
	\item Europe , gaz = Russie
\end{itemize} 

Ces réserves ont évolués dans le temps. Trois types d'entités déclarent leurs réserves : sociétés cotés en bourses, petites sociétés non cotés en bourse et États.
Depuis 1980, il y a une multiplication par 2.75 des réserves prouvés de pétrole et de gaz.
L'augmentation des réserves est du a deux pays  : 
\begin{itemize}
	\item Venezuela : grandes réserves mais de mauvaise qualité (donc ne risque pas de bouleverse le marché du pétrole)
	\item Canada : Sables bitumineux
\end{itemize}

\subsection{pollutions associées aux énergies fossiles}
\textbf{Extractions}
\begin{itemize}
	\item Pollutions locales
	\item Rejets de méthane
\end{itemize}
\textbf{Raffinage}
\begin{itemize}
	\item Rejets de gaz et de liquides dans l'environnement
\end{itemize}
\textbf{Combustion}
\begin{itemize}
	\item Rejet de chaleur
	\item Pollutions sonores
	\item Émission de polluants
	\begin{itemize}
	[Pollutions locales]
		\item CO : monoxyde de carbone
		\item SO2 : dioxyde de soufre
		\item NOX : oxyde d'azote
		\item Composés organiques volatiles
		\item PM 2.5
	[pollutions globales]
		\item CO2 : dioxyde de carbone
		\item CH4 : méthane
	\end{itemize}
\end{itemize}
\textbf{Stockage et transport }
\begin{itemize}
	\item Charbon : poussières et lessivage
	\item Pétrole et gaz naturel : fuites
\end{itemize}

\subsection{Le concept du pic pétrolier}
\subsubsection{Le rôle du pétrole}
Utilisation des énérgies fossiles : le charbon \\
On remarque que le charbon est en majorité utilisé par l'industrie. \\
Le charbon représente la majorité du mix électrique chinois. En 2018, la contribution de la chine en consommation de charbon est de 64\%.

Utilisation des énergies fossiles : le gaz \\
Le gaz est en grande partie utilisé dans l'industrie  et le logement (donc diversification des utilisations comparé au charbon) \\

Utilisation des énergies fossiles : le pétrole  \\
En majorité utilisé par les transports. Il n'y a pas d'alternative au pétrole pour le moment.\\

\begin{enumerate}
	\item Le marché du pétrole initial : pétrole lampant (kérosène), petit marché. \\
	\item Lubrifiant pour les machines 
	\item LE mazout pour le transport maritime
	\item Après la WWI : essence et kérosène pour les transports routiers et aériens
	\item Après la WWII : pétrochimie
\end{enumerate}
$\Rightarrow$ prix adapté aux différents marchés pertinents  \\

Une industrie fortement capitaliste
\begin{itemize}
	\item Phase d'extraction : Couts Fixes très forts et risques importants 
	\item Phases de raffinage et transports : couts fixes forts
\end{itemize} 

Une industrie avec forte rentes différentielles  (accès a une ressources plus profitable que celle des concurrents)
\begin{itemize}
	\item Deux caractéristiques : densité et teneur en impuretés (soufre)
\end{itemize}

Hétérogénéité des couts de production. Due a la technique (extractions, raffinement)
\subsubsection{Le pic pétrolier}
Dr Hubert prédit en 1959 un pic de production de pétrole aux USA en 1970. \\
La production d'un gisement suit une courbe de Gauss. \\
Cette méthode a été appliquée a l'échelle mondiale, $\rightarrow$ pic en 2006. \\
Les données de réserve : Attribuer les quotas de vente en fonction des réserves déclarés. Certains pays ont gonflé artificiellement leurs réserves .

\textbf{La consommation :}  \\
En occident, la consommation stagne alors que en chine et en Inde. 
Le budget carbone : budget qu'on décidé de s'allouer les états,  50\% de chance de rester a 2 degrès.
\begin{itemize}
	\item les actifs échoués : Pays qui ont investi des capitaux dans l'énergie pétrolière, si transition énergétique $\Rightarrow$ Perte d'actifs
	\item Paradoxe Vert : Si des pays décident d'une transition énergétique, les pays vendeurs de pétrole diminuent le prix de pétrole et tout le monde recommence a consommer du pétrole
	\item Carbone capture storage : Produire de l'électricité avec des énergies fossiles sans polluer
\end{itemize}

\subsubsection{Le processus de changement climatique}
\textbf{Le cycle du carbone} : Circulation du carbone entre 4 réservoirs ( Lithosphere, atmosphère, Hydrosphère, Biosphère).Il y a une accumulation qui se fait dans l'atmosphère . \\
\textbf{Effet de serre} : s'est renforcé et fait augmenter la chaleur.\\
Les différents effet de serre : vapeur d'eau (durée de vie de 24h). \\
PRG pouvoir de réchauffement global permet de comparer les différents gaz a effet de serre (CO2, CH4, N2O, HFC). 
Lien entre la concentration de gaz a effet de serre et chaleur. \\
Le rensenti est très hétérogène, il n'aura pas d'évolution linéaire, il y aura un rattrapage des zones moins chauffées.. \\

\textbf{Les causes naturelles et anthropiques de l'effet de serre} : \\
Causes naturelles :  sur de très très longues périodes, la température peut changer. Or la cause actuelle du réchauffement climatique ne peut pas etre expliqués sans causes anthropiques.\\

\textbf{Les rétroactions positives} \\
Effet de boule de neige (phénomène qui s'auto alimente)
\begin{itemize}
	\item Effet albédo : Fonte des neiges blanche $\Rightarrow$ remplacées par plus delements sombres qui absorbent plus de chaleur (océans)
	\begin{itemize}
		\item	Fontes des neiges /glaces
		\item Dinimution de l
	\end{itemize}
	\item Fonte du permafrost
	\begin{itemize}
		\item Fonte initialement attendue pour 2090
		\item Grandes quantités de méthane sous le permafrost (changement climatique)
	\end{itemize}
	\item Acidification des océans et plus faible dissolution du CO2
	\begin{itemize}
		\item Perte de la capacité d'absorber le CO2 par l'eau
	\end{itemize}
	\item Augmentation de la respiration des sols
	\begin{itemize}
		\item Champignons respirent etc ...
	\end{itemize}
	\item Saturation de la fonction de puit de carbone des forêts tropicales.
\end{itemize}
$\Rightarrow$ Amplifie l'inertie du systeme climatique.

\textbf{Intertie du systeme climatique} : 
L'inertie la plus forte est celle de la montée des eaux .
\textbf{LEs pricnipales conséquences du réchauffement climatique}
\begin{itemize}
	\item Conséquence 1 : Augmentation de la température moyenne
	\begin{itemize}
		\item Deja observée depuis le début des 90s, plus forte sur les contentents en Arctique.
		\item Hétérogénéité des températures , donc probleme de dimensionnement du systeme électrique
		\item Hausse de la moyenne et variabilité
	\end{itemize}
	\item Conséquence 2 : Précipitations et 
	\begin{itemize}
		\item Hausse des précipitations : bande équatoriale et arctique 
	\end{itemize}	 
	\item Conséquence 3 : atteinte des éco systèmes 
	\item Conséquence 4 : Elevation des océans 
\end{itemize}

\textbf{Impacts régionaux à 2050}
Tous les pays vont adopter plusieures stratégies 
Europe : problèmes en rapport avec l'eau
\begin{itemize}
	\item Innondations
	\item Evenements climatiques extremes
	\item Réduction de la disponibilité d'eau
\end{itemize}
Asie : Ne subit pas encore le réchauffement
\begin{itemize}
 \item Pénuries d'eau et malnutrition (court terme)
 \item Inondations 
 \item Mortalité liée à la chaleur
\end{itemize}
Amérique du nord
\begin{itemize}
	\item Incendies
	\item Mortalité liée a la chaleur (long terme)
	\item Inondations
\end{itemize}

\subsubsection{Les négociations internationales sur le climat}
GIEC : négociations entre scientifiques.
GIEC donne les objectif carbone, mais pas de scénarios

\subsubsection{Les COP conférences of parties}
caractéristiques 
\begin{itemize}
	\item Cycles de négociations
	\item Organisation en sessions parallèles thématiques (plusieurs groupes de débat), certaines sessions perdent en importance au cours du temps.
	\item Négociations caractérisées par une forte incertitude quand aux résultats.
	\item Principaux résultats.
	\begin{itemize}
		\item Responsabilité commune mais différenciée 
		\item protocole de Kyoto
		\item Mécanisme de flexibilité (prendre un engagement a réduire les émissions, mais avec une certaine flexibilité = répartition de l'effort) (système d'échange des droit a émission) (mécanisme de dev propre )
		\item COP21 $\Rightarrow$ Accord de Paris
	\end{itemize}
\end{itemize}

Coalitions de la COP 
\begin{itemize}
	\item Petites Iles ont extrement de poids (car un pays = une voix)
	\item G77 & Chine
	\item  
\end{itemize}

Il y a différence entre stocks et flux d'émissions de CO2 

Il existe une relation direte entre niveau de développement économique et émission par tête. 
Pour les pays en voix de développement il y a un grand cout d'opportunité a mettre en place les décisions de la COP21. Cest pays là préfèrent donc continuer sur les energies non renouvelable pour pouvoir croitre.\\

Le commerce international est un vecteur à l'emission de gaz a effet de serre.

Conferences of parties : principaux cycles 
Cop 1 $\rightarrow$ Cop 11 \\
2000 : rupture des négociations 
2001 : sortie des US du protocole de Kyoto.
2004 : La russie signe le protocole de Kyoto $\Rightarrow$ le protocole de Kyoto rentre en compte 
2005 : Cop11  = mise en place du protocole de Kyoto.
On anticipe que la cop de copenhague sera un succès, il y a grandes anticipations (grande couverture médiatique).
Cop 11 $\rightarrow$ Cop 15 : echec de Copenhague \\
En 2009, les négociations repartent a 0. De plus la crise économique pousse les états a ne pas acter sur un accord. 
L'accord de paris est construit sur l'échec de celui de copenhague qui avait pour stratégie de mettre en place des accords contraignants. L'accord de Paris : ne pas dépasser deux degrès, pas d'accord contraignants mais révisions a la hausse des engagements tous les 5 ans. 
Cop 15 $\rightarrow$ Cop 21 : accord de Paris \\
Cop 21 $\rightarrow$ Cop 25 : echec de la COP de madrid \\

\subsection{La tarification des émissions de GES}

Quels sont les avantages de la taxe pigouvienne ?
La faiblesse des coûts de mise en place, relativement à d'autres instruments 
\begin{itemize}
	\item LE signal prix est une alternative aux instruments "command and controlé
\end{itemize}

Le revenu fiscal peut generer du double dividende : On met une taxe qui permet d'ameliorer la situation (premier dividende), si on est un état on touche un revenu (second dividende)
Certaines taxes distortives, génèrent un poids mort dans l'économie (ex : taxe sur les prix). Toutes les taxes ne sont pas distortives (ex taxe sur le profit).
Les économistes proposent de supprimer les taxes distortives au profit de taxes écologiques.

Le giec propose de baisser la fiscalité sur le travail

\subsection{Les limites de la taxe pigouvienne}
A court terme, on peut rétablir l'optimalité mais pas a long terme. \\
A court terme, le cout de prod va augmenter > augmentation du prix > diminution de la qt totale (baisse demande).
avant taxe : qmin (minimisation du cout)
Après taxe : q*

A long terme : des entreprises peuvent sortir (ou rentrer), la pression concurrentielle permettra aux entreprises de minimiser leurs couts et revenir a la quantité de minimisation des coûts.
La taxe permet de controler l'output total, le nombre de firmes , mais pas la production d'une entreprise particulière 

Risque de concentration plus forte sur le marché (générer un oligopole/monopole)
Il faudrait coupler la taxe pigouvienne avec impots/subventions sur les entreprises qui sortent et entrent (pour gérer N)

Critique : 
La taxe pigouvienne est déduite en fonction de la condition d'optimalité en cpp

Réponse a la critique : La taxe pigouvienne optimale est impactée par la valeur du taux d'actualisation.


Alternative a la taxe : 
pas un probleme de signal prix.
Dès que des individus partagent un bien commun, chacun des individus va extraire plus que ce qui est optimal.
Sur les ressources naturelles qui subissent des dégradations, son propriétaire reçoit une compensation.
Il faut donc faire respecter les droits de propriété.

Limites

qui est propriétaire du systeme climatique ?
Nécessite une information parfaite sur les conséquences des pollutions
Hold up (monopolisation)

Cap and trade 
Plutot que mettre un prix, on met un objectif de pollution et on laisse les gens échanger les permis de pollution.
Le choix du cap est un compromis politique.
Les entreprises pour
Systeme d'échange de quota : seul secteurs : minéraux énérgies , métaux, pate a papier.
 
phase 2 : lancement du protocole de kyotot
ditribués : via gound featherin
ceux qui polluent le plus ont eu le plus de permis

\end{document}
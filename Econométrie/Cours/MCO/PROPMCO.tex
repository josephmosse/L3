\documentclass{article}
\usepackage[utf8]{inputenc}
\usepackage{multicol}
\usepackage{array}
\usepackage{upgreek}
\usepackage{amsmath}
\usepackage{physics}

\usepackage[dvipsnames]{xcolor}
\usepackage{geometry}

\title{Propriétés des estimateurs des MCO}



\begin{document}
\newcommand{\halpha}{\hat{\upalpha}}
\newcommand{\hbeta}{\hat{\upbeta}}
\newcommand{\sumt}{\sum_t}
\newcommand{\epsilont}{\varepsilon_t}
\newcommand{\wt}{w_t}
\newcommand{\sig}{\upsigma_\varepsilon^2}
\newcommand{\xt}{x_t}
\n

\begin{align*}
\halpha
\hbeta
\sumt
\epsilont
\wt
\sig
\end{align*}


\maketitle
\tableofcontents
\newpage

\section{Les estimateurs des MCO sont des fonctions linéaires de \(Y_t\) de \(Y\)}

\section{\(\halpha\) et \(\hbeta\) sont des estimateurs sans biais}

\section{\(\halpha\) et \(\hbeta\) sont des estimateurs à variance minimale de \(\upalpha\) et \(\upbeta\)}

\section{\(\halpha\) et \(\hbeta\) sont des estimateurs convergents}

\section{Détermination d'un estimateur sans biais de \(\sigma_\varepsilon^2\)}

\section{Les estimateurs des moindres carrés \(\halpha\) et \(\hbeta\) de \(\upalpha\) et \(\upbeta\) correspondant au maximum de vraisemblance}

\end{document}
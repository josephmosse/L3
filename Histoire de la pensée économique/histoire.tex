\documentclass{article}
\usepackage[utf8]{inputenc}
\usepackage{multicol}
\usepackage{tikz}

\title{Histoire de la pensée économique}

\begin{document}
\maketitle
\tableofcontents
\newpage
Comme les champs étaient très imbriqués ,les agriculteurs devaient synchroniser leurs cultures (par périodes/cycles).

\section{L'économie politique classique}

\subsection{Les précurseurs}
Les précurseurs font parti du contexte intellectuel duquel va émerger l'économie classique

\subsubsection{Bernard de mandeville}
Bernard de Mandeville est un personnage important du paysage intellectuel du \(18^{\textrm{ème}}\) siècle dans lequel se situe l'économie politique classique. Il est connu pour avoir écrit "La fable des abeilles ou les vices privés et les bénéfices publiques".

\begin{itemize}
	\item 1703 : traduction en Anglais des fables de la Fontaine
	\item 1705 : Court poème  : the grumbling hive ; le message est assez controversé  (\(\Rightarrow\) Succès de l'ouvrage)
\end{itemize}

Tout l'objet de ce qu'écrit Mandeville est de savoir comment des individus qui ne sont pas des gens biens (qui ont des comportements vicieux) peuvent engendrer des conséquences positives par ces comportements (comportements mauvais \(\Rightarrow\) conséquences positives) (Comment le vice conduit au bien public)

Son oeuvre est dirigé contre un philosophe Anthony Cooper (1671- 1713) Shaftesbury a influencé Adam Smith et David Young.

Pour Shaftsebury il existe une harmonie naturelle entre les humains dans laquelle le bien commun est atteint et dans laquelle le bien et le beau coïncident . Ceci est rendu possible car il existe un sens moral chez les humains qui est inné. Ce sens moral inné doit être entretenu par l'éducation.

Mandeville s'oppose a Shaftsebury, selon lui ce qui guide  l'être humain est : l'intérêt et les passions. Donc tout l'ordre social repose sur l'intérêt personnel et donc sa recherche conduit au bien collectif. Les individus ne cherchent pas à faire le bien commun, ils cherchent a agir pour leur propre bien. Le bien commun est donc la conséquence (non recherchée) de la recherche de l'intérêt privé.

Histoire : Mandeville cherche a expliquer quelque chose a propos des sociétés humaines \\
Une vaste ruche bien fournie d'abeille, elle est riche. Chacun poursuit ses activés. Chaque partie étant pleine de vice, cependant le tout est un paradis

Pour Mandeville le comportement vicieux est poursuivre son intérêt personnel. \\
Mandeville soutien la monarchie (La tyrannie et la démocratie sont mauvaises). Les rois gouvernent et sont contraints par les lois. L'intérêt personnel mène donc au bien commun lorsque des règles sont bien établies avant.

Pour Mandeville le laissez faire conduit a des effets positifs seulement lorsque un politicien (monarque) est assez compétent pour guider l'intérêt personnel vers le bien commun.

Le corollaire est que si les individus cherchent a faire le bien commun intentionnellement; alors les conséquences seront négatives. La ruche commence a décliner lorsque les abeilles demandent a Jupiter de changer leur comportement individuel pour un comportement vertueux \\

Planifier le fonctionnement d'une économie ne peut pas véritablement fonctionner (trop d'individus) et a l'inverse, les gens doivent se mettre d'accord entre eux, et étant dans leur  les gens doivent respecter ces accords.

Il n'y a jamais de crise économique durable , les crises économiques se résorbent d'elle même. Cela veut dire que les marchés s'auto-régulent.

La différence avec Smith (main invisible) : Pour smith, la recherche de l'intérêt privé produit de la richesse uniquement sur les marchés, en dehors des marchés, Smith croyait a la morale et a l'éthique (Sympathy $\cong$ empathie). Alors que pour Mandeville la recherche de l'intérêt privé mène toute la société.

Mandeville défend l'important du luxe ( permet de maintenir l'économie dans une situation de croissance) ($\cong$ théorie du ruissellement). Il défend aussi l'importance des dépenses.

Citation incendie de Londres : Grande calamité, les coûts ont été énormes, mais si l'on met dans la colonne des bénéfices tous ceux qui ont bénéficié de la reconstruction de la ville; les réjouissances égaleraient les plaintes. Une calamité engendre des bénéfices). Il faut toujours raisonner entre coût et bénéfices.
Il ne faut pas résonner ceteris paribus.
Cette citation illustre l'importance de l'intérêt personnel; si chaque individu dans chaque profession poursuit son intérêt personnel, alors la somme des gains va être suffisante pour compenser le coût de l'incendie. A l'inverse, si l'on veut faire le bien , alors il faut éviter l'incendie, on se prive alors de tous les bénéfices.
\\

\begin{itemize}
	\item Toute action a des effets positifs et des aspects négatifs
	\item Le raisonnement ne peut pas être ceteris paribus
\end{itemize}

\subsubsection{David Hume}

Importance de la sympathie, (importance de la prise en compte d'autrui dans son comportement) et interet personnel (prise en compte de soi même)  Bipolarité entre les deux \\
Relativisme : chaque individu a sa propre perception du monde et que toutes visions cohabitent ensemble (problème du libéralisme). \\
Hume rajoute a cela un autre niveau : \textbf{Le droit de propriété}.\\
Il faut des institutions, des conventions qui établissent des droits de propriété. \\
Sur ces bases là l'économie classique tire ses origines \\

\subsection{Adam Smith, L'invention de l'économie classique}
Adam smith (1723 - 1790) Enseignant a l'université d'Edeimburgh. Smith n'a pu observer de la RI que ses premiers moments (que les moments où elle a été positive). Cela explique \textbf{l'optimisme} de Smith en l'ordre spontané, le marché libre. \\
Smith fait parti du même courant que Hume, opposé au rationalisme cartésien. Il était philosophe aussi, intéressé dans l'astronomie (qui influenceront son analyse economique). \\
Ouvrages majeurs de Smith : 
\begin{itemize}
	\item La théorie des sentiments moraux - 1759
	\item Recherche sur la nature et les causes de la richesse des nations - 1776 (naissance de l'économie politique classique)
\end{itemize}
Lors d'un voyage sur le continent Européen, Smith rencontre Quesnay, et restera en lui une par de vision physiocrate. \\
Un élément central chez Smith : \textbf{La propension a l'échanges} = caractéristique propre aux humains (conception de l'individu selon Smith). \\
Toute l'économie politique de Smith repose sur l'échange. L'échange peut être motivé soit par \textbf{l'intérêt personnel} ou soit par \textbf{la sympathie pour autrui }. Pour échanger, il faut mettre en place des instituions, le marché est le lieu sur lequel on effectue les échanges. La richesse des nations va donc dépendre de la capacité a échanger et des marchés créés pour l'échange \\
Plus les marchés sont grands, plus il y aura d'échanges et donc plus de richesse (cercle vertueux) \textbf{La richesse des nations dépend de la taille des marchés}. \\
Dans ce processus, un élément important se dégage : la \textbf{la division du travail}, celle ci permet d'augmenter la taille des marchés (plus le travail est divisé, plus les marchés sont larges et donc plus la nation est riche). \\
A l'inverse dans un petit marché, il n'y a peu d'incitation a se spécialiser/diviser le travail et si il n'y a pas de spécialisation, il n'y a pas d'efficacité (on ne peut créer de la richesse que si l'on est efficace)\\

\subsubsection{Le problème Adam Smith (Das A.S Problem - Hegel)}
Les lecteurs de Smith pensent qu'il y a un différence majeure entre les deux ouvrages de Smith. Dans le premier Smith semble mettre en avant deux motivations a échanger contradictoires, dans la TSM la morale découle des sentiments, l'une des motivation essentielle que les individus ont a agir est la recherche de sympathie. Chez Smith comparé a Hume, la sympathie est un sentiment (que l'on éprouve pour les autre) "altruisme", les individus agissent en prenant compte le bien être des autres (besoin d'obtenir l'approbation des autres) ce qui conduit a des actions non conflictuelles (homogénéisation des passions et sentiments), la sympathie permet a chacun les sentiments que les gens éprouvent. \\

La sympathie est universelle (chaque humain a cette capacité), mais elle diminue avec la distance. Exemple : Est ce qu'un individu en Europe va sympathiser avec des victimes d'une catastrophe en Chine ? $\Rightarrow$ Pour Smith non car il y a une idée de proximité dans la sympathie. \\

Dans la richesse des nations, Smith change de perspective ( \textbf{Ce n'est pas de la bienveillance} du boucher etc ...). La sympathie n'a aucun rôle a jouer dans les échanges marchants, l'intérêt personnel en est le principal. \\

Problème Adam Smith: contradiction entre les échanges marchants qui sont guidés par l'intérêt personnel et les échanges non marchants guidés par la sympathie et la bienveillance. La conclusion tirée est que Smith est contradictoire , cette opposition ne peut pas être résolue. Cette contradiction entre les deux textes va durer jusqu'aux années 80. \\

Comment résoudre ce problème ? \\
\begin{itemize}
	\item 	La sympathie est le sentiment qui guide les relations personnelles, qui sont en général non marchands . Sur les marchés, la sympathie n'existe pas car les relations sont impersonnelles  (puisqu'il n'y a plus de relation personnelles, l'intérêt personnel prend le relai.
	\item Il y a une distinction stricte entre les comportements marchands et non marchands. Cette explication est difficilement acceptable car : dans les relations personnelles des éléments marchands rentrent en ligne de compte et dans les comportements marchands , des sentiments personnels peuvent jouer
	\item L'intérêt personnel conduit a la sympathie. L'intérêt personnel chez Smith renvoi a l'amour de soi (self-love) qui est une forme de sympathie (pour soi même). Le problème Adam Smith n'existe pas, l'intérêt personnel et la sympathie se complètent plus qu'ils ne s'opposent
\end{itemize}

\subsubsection{Valeur utilité ou valeur travail}
Pour pouvoir échanger il faut se mettre d'accord sur une valeur commune (aux individus impliqués dans l'échange). Smith est le premier à introduire la différence entre valeur d'usage et valeur d'échange. \\
Le mot valeur a deux significations, il peut signifier l'utilité qu'a un objet pour l'individu et peut désigner la capacité de cet objet a acheter d'autres objets (la première correspond a la valeur d'usage, la seconde, celle d'échange). La valeur d'usage est mesurée en satisfaction alors que la valeur d'échange est mesurée en prix. Il y a des objets qui ont une très forte valeur d'usage mais aucune valeur d'échange et réciproquement. Donc l'utilité ne joue aucun rôle secondaire dans la détermination de la valeur d'échange.\\
Smith s'oppose a l'idée que la valeur d'échange soit fondée dans l'utilité et pour illustrer ce propos, il utilise la métaphore du paradoxe de l'eau et du diamant (il n'y a rien de plus utile que l'eau, mais avec elle on ne peut presque rien acheter, au contraire, un diamant n'a presque aucune valeur quant a l'usage , mais on retrouvera fréquemment a l'échanger contre une grande quantité de marchandise). \\
On ne peut pas fonder la valeur d'échange dans l'utilité, donc le prix n'est pas fondé dans l'utilité . Sur quoi fonde t-on la valeur d'échange ? \\
Pour smith le travail et la mesure réelle de la valeur d'échange des marchandises, il introduit la théorie de la valeur travail.\\
2 conceptions fondamentales
\begin{itemize}
	\item La théorie de la valeur travail commandée = quantité de travail qu'un objet peut acheter (la valeur d'un bien dépend de la quantité de travail qu'il peut acheter)
	\item La théorie de la valeur travail incorporée dit que la valeur d'échange d'un bien comprend la quantité de travail necessaire pour frabriquer le bien
\end{itemize}

rattraper cours 04/10

3eme type de revenu : salaire. C'est le revenu fondamental car le travail est la source des richesses. Le salaire est important car il détermine la part des autres revenus. Le salaire est le prix d'une marchandise (ce qui est différent des deux autres revenus, rente et profit). Le salaire peut prendre deux valeurs : une valeur naturelle et une valeur de marché.
\begin{itemize}
	\item[Le prix naturel/ salaire naturel] : est déterminé par les coûts de production d'une unité de travail. Somme des salaires rente et profit qui ont été dépensés pour produire du travail (ou une unité de travail) = tout ce que coute le fait de faire vivre un travailleur. Le salaire naturel est donc le salaire de subsistance. Le salaire de subsistance dépend de l'environnement culturel dans lequel on est.
	\item[Le salaire courant/salaire de marché] est déterminé sur le marché du travail et qui dépend des conditions qui existent sur ce marché (O&D). Smith ne considère pas que le marché du travail soit concurrentiel $\Rightarrow$ le salaire est détermine par une négociation entre travailleurs et employeurs . \\
	Il s'applique la loi de gravitation du salaire courant autour du salaire naturel. Le salaire courant est rarement égal au salaire subsistant. Si le salaire courant est supérieur au salaire naturel, le niveau de vie augmente et la l'offre de travail baisse et inversement. Il y a donc un mécanisme de régulation qui fait que le salaire de marché tend a etre egal a celui naturel.
\end{itemize}
Ferdinent Lassalle , d'airain des salaires : les travailleurs ne peuvent pas gagner plus que le salaire de subsistance. Le salaire de marché va etre sup a celui naturel si l'économie est en croissance . Quand l'économie est en croissance, l'augmentation de demande de marchandise entraine une augmentation de la demande de travail et donc une hausse des salaires. Pour Smith, en situation de croissance, le marché protège les travailleurs. Plus une économie est riche, plus elle peut payer ses travailleurs, et donc plus elle va les payer.
Cet argument est un argument qui le pousse a défendre la croissance économique.
Dans le long terme, les profits ont tendance a baisser, les possibilités d'investissement rentable sont de moins en moins importants. Docn les économies capiltaistes tendent vers une situation dans laquelle il n'y a plus de croissance. Sans croissance les salaires sont réduit au minimum vital

\subsection{Commentaire}
Smith est le premier a dinstinguer "valeur utilité" et "valeur d'échange"
Il y a deux définitions possibles de la valeur. Pas besoin de dire que c'est une citation d'adam smith. Qu'est ce qui fait qu'un bien a de la valeur ? Il faut distinguer deux choses , le prix (valeur que le bien prend dans l'échange) et l'utilité (valeur que le bien a pour l'individu. On parlera de Smith car il est le premier a avoir introduit la distinction entre les deux formes de valeur (citer Smith de manière subtile).
Il faut rappeler que la valeur d'échange est le prix. Il est aussi possible que la valeur d'usage ne permette pas de déterminer la valeur d'échange.
Si c'est le cas, deux questions : 
\begin{enumerate}
	\item Pourquoi ? comment peut on le justifier
	\item D'où viens la valeur d'échange. Si elle n'est pas fondée dans l'utilité, elle est fondée dans le travail
	\begin{enumerate}
		\item Valeur travail commandée : Smith / Malthus
		\item Valeur travail incorporée : Ricardo
	\end{enumerate}
\end{enumerate}
Si la d'échange est obtenu grace a la valeur d'usage. La question que se posent les économistes classiques est la valeur d'échange. Pour des économistes qui sont intéresses dans la compréhension de l'échange (comme les classiques), il est crucial de comprendre comment se fixent les taux d'échanges entre les produits lorsque les individus s'engagent dans des transactions marchandes. ces taxu d'échange, comme le dit l'auteur de cette citation "la faculté que donne un objet d'acheter d'autres marchandises peut être déterminé de plusieures façons. Dans l'hpe, il existe au moins deux grandes théories qui expliquent comment se determinent les prix : les théories fondées sur l'utilité et les théories fondées sur le travail. \\
Plans possibles : 
\begin{enumerate}
	\item Les théories de la valeur travail
	\item Les théories de la valeur utilité
\end{enumerate}

 Les économistes classiques ont proposé une théorie de la valeur d'échange sur le travail, cela veut donc dire qu'elle n'est pas basée sur l'utilité
\begin{enumerate}
	\item Pourquoi les valeurs sont différentes
	\item En quoi consiste une théorie de la valeur travail
	\item Est il vrai que le postulat des classiques celon lq la valeur d'usage et d'échange diffèrent peut etre encore accepté
\end{enumerate}

On choisi le second plan 
\subsection{Pourquoi les valeurs sont diff}
Les classiques au premier rang desquels se trouve Smith , ont fait la différence des deux valeurs, car selon eux il existe des biens qui ont une valeur d'usage (une utilité pour les individus), mais aucune valeur d'échange et inversement. C'est le paradoxe de l'eau et du diamant (Smith) mais aussi chez Ricardo qui commence son traité d'économie politique et de l'impôt en citant Smith (paradoxe eau et diamant) tout en utilisant d'autres biens. \\
On pourrait rajouter un argument qui est que toutes les marchandises ont plusieurs usages (donc utilités), cependant le prix ne diffère pas, ce qui renforce l'argument des classiques qui est que on ne peut pas baser le prix sur l'utilité , il faut un critère objectif qui permette de comparer les valeurs des différents biens et donc de les échanger, pour les classique ce critère objectif est la valeur travail.
Les théories de la valeur travail : 2 façons de la concevoir , la première consiste a dire que le prix dépend de la qt de travail que ce bien peut acheter (valeur travail commandée). Et la seconde qui consiste a dire que le prix dépend de la quantité de travail dépensée pour creer le bien (valeur travail incorporée).
Smith considérait que les deux théories étaient confondues dans les sociétés pré capitalistes (vtc = vto), mais plus le cas dans les sociétés capitalistes. Il faut utiliser une théorie de la valeur travail commandée.
Malthus était d'accord avec Smith, mais Ricardo totalement opposé a Smith sur ce point. PEut on dépasser cette opposition, -oui , c'est ce que proposent les marginalistes (théorie de la valeur utilité).


\subsection{Malthus}
Malthus a relié l'augmentation de la population et la malnutrition (pour lui, si la population est malnourrit, parce qu'ils sont trop nombreux). Sur la base de cette intuition Malthus écrit (1798) essai sur le principe de population (qu'il ne signera pas) Malthus se situe en opposition a Godwin. il signera son édition a partir de 1803, et publiera 6 éditions pendant sa vie.
Thèses de l'ouvrage : 
\begin{itemize}
	\item la population augmente selon le rythme d'une progression géométrique et double a peu près tous les 25 ans
	\item la subsitance augmente de façon arithmétique, a partir de la seconde période, les subsistances sont devenu moins grande que la population.
\end{itemize}
Il se base sur les données empiriques de la population en amérique, or la population en amérique est en partie due a l'immigration et non aux naissances. Malthus avait une crainte a propos des subsistances car en angleterre, il y a un problème de subsitance car la population augmente, car étant une île, la quantité de terre disponible est limitée. Malthus savait que cette affirmation n'était pas réaliste, il savait que le mouvement représenté par ces courbes n'allait pas se produire. Malthus met en evidence cette evolution en sachant que ce n'est pas rélasite pour montrer ce qu'il risque de se produire si on empêche les mécanismes de contôle de la population de se mettre en place et de jouer. \\
Il n'envisage pas de mécanismes qui induisent une augmentation des subsistances (car il croît en les rendements décroissants ), il y a une limite a la quantité de nourriture que l'on peut produire 
Deux moyens de contrôle (préventifs et positifs):
\begin{itemize}
	\itme Mécanismes préventifs : mécanismes qui préviennent les naissances
	\begin{itemize}
		\item Capacité des individus a anticiper l'impossibilité de nourrir de grande familles (rationalité)
		\item Moral restrain : restriction morales, l'ensemble des facteurs psychologiques qui vont inciter les individus a ne pas faire d'enfants. Pour Malthus, cette contrainte morale doit être mise en œuvre. Malthus était contre les lois d'aide aux pauvres, car elles sont un moyen de permettre l'augmentation de la population.
	\end{itemize}
\end{itemize}

Malthus a influencé des théologistes, mais aussi Darwin (1859, de l'origine des espèces) en 1838, Darwin a retenu deux éléments de Malthus *
\begin{itemize}
\item le principe de séléction : une partie de l'ensemble des individus nés ne va pas survivre, le corolaire de cette idée est que les espèces font plus de décendants que ce qu'il peut en survivre 
\item Principe de survie : Quand il y a trop d'enfants qui naissent, il y a une compétition entre eux et seul les mieux adaptés survivent (survival of the fittest - Herbert spencer) 
\end{itemize}
Le principe de la selection naturelle a aussi été inventé par Wallace .
Darwin n'a pas que lu Malthus, il a aussi lu Adam Smith (TSM).

\subsubsection{Les idées économiques}
1ER pt : Malthus adopte une théorie de la valeur (travail), il suit a Adam Smith (meme distinction entre valeur travail incorporée et commandée) et s'oppose a une théorie de la valeur travail incorporée (pour la meme raison) Le prix de valeur d'échange des marchandises inclu des éléments qui ne sont pas reliés au travail : la rente et le profit, par conséquent le prix des marchandises ne peut pas être proportionnel au travail dépensé (travail incorporé dans la marchandise), "la valeur des marchandises a toujours pour mesure la quantité de travail ordinaire qu'elle peut rétribuer". Malthus estime qu'il existe un prix naturel et un prix de marché, que le prix naturel est le prix nécaissaire pour couvrir les couts de production, avec pour différence : Malthus a une théorie pour expliquer pourquoi le taux de salaire de marché ne peut pas etre différent du taux de salaire naturel . Le rôle de l'offre et de la demande est plus important chez Malthus que chez Smith.

2nd pt : La théorie de la rente
Il met en avance deux points: le rôle de la fertilité des terres et le fait que la rente est un revenu différentiel. 
Malthus théorise les rendements décroissants. Plus la population augmente plus on utilise de terres (moins fertiles ) moins les rentes sont importantes .
La rente est obtenue par différences en fonction des terres . La théorie de la rente que MAlthus esquisse va etre par la suite développée par Ricardo

3eme pt : Demande effective et crises . L'offre de produit est une demande simultanée de produits cad : chaque fois qu'on offre des biens, on crée une demande. Si l'offre crée une demande équivalente, il ne peut pas y avoir de crise économique générale et durable. On retrouve avec Say que les marchés s'autorégulent et que les crises se résorbent d'elle mêmes. Ricardo défend la même idée (constante chez les classiques). Malthus est en désaccord avec cette théorie.
Malthus s'oppose a cette idée d'auto-régulation, la demande est faite par ceux qui ont les moyens et la volonté de demander (pour say demande effective = demande potentielle, l'épargne ne représente pas une disparition de ressources, il n'y a pas de thesaurisation).
Pour Malthus, l'épargne va etre trop importante lorsque la consommation est insuffisante et donc la demande va etre insuffisante et l'offre plus forte que la demande , donc crise.
Malthus dit que pour qu'il y ait une offre suffisante pour absorber l'offre de produire, il faut que les individus aient un revenu suffisant pour demander. Il faut que les travailleurs aient les moyens d'acheter des biens. SI les travailleurs n'ont pas de revenus suffisants , il y a une crise de surproduction.
Malthus fait quelques recommandations de politiques economiques : 
Il faut entretenir les consommateurs improductifs  (permettent de réduire l'écart entre l'offre et la demande)
Suggère de diviser la propriété foncière pour diminuer la richesse moyenne par individus car la propension a consommer diminue avec le revenu. Quelques historiens soulignent que Malthus a développé une théorie de la demande effective qui anticipe celle de Keynes. Malthus est le seul qui croit aux crises de surproduction
\end{document}





